% Sample LaTeX file for creating a paper in the Morgan Kaufmannn two
% column, 8 1/2 by 11 inch proceedings format.

\documentclass[letterpaper]{article}
\usepackage{uai2018}
\usepackage[margin=1in]{geometry}

% Set the typeface to Times Roman
\usepackage{times}

\title{structure-random-projection}

\author{} % LEAVE BLANK FOR ORIGINAL SUBMISSION.
          % UAI  reviewing is double-blind.

% The author names and affiliations should appear only in the accepted paper.
%
%\author{ {\bf Harry Q.~Bovik\thanks{Footnote for author to give an
%alternate address.}} \\
%Computer Science Dept. \\
%Cranberry University\\
%Pittsburgh, PA 15213 \\
%\And
%{\bf Coauthor}  \\
%Affiliation          \\
%Address \\
%\And
%{\bf Coauthor}   \\
%Affiliation \\
%Address    \\
%(if needed)\\
%}


%\usepackage{amsmath,amsbsy,amsgen,amscd,amssymb,amsthm,bm,xspace}
\usepackage{graphicx}
\usepackage{setspace} % for single/double spacing

% for tensors
\usepackage{stmaryrd}
\usepackage[mathscr]{euscript}

% \usepackage{subfig}
% \usepackage{float}
% \usepackage{caption} % causes weird compile problem?
% \usepackage[font=small,margin=0.25in,labelfont={sc},labelsep={colon}]{caption}
% \usepackage{subcaption}

\usepackage{thmtools}
\usepackage{thm-restate}
\theoremstyle{definition}
\declaretheorem[name=Theorem,numberwithin=section]{thm}
\declaretheorem[name=Lemma,numberwithin=section]{lem}
\declaretheorem[name=Assumption,numberwithin=section]{assumption}

% algorithms
\usepackage[noend]{algpseudocode}
\usepackage{algorithm,algorithmicx}
\algrenewcommand\alglinenumber[1]{\sf\tiny\color{medblue}{#1}\quad}
\algrenewcommand\algorithmicrequire{\textbf{Input:}}
\algrenewcommand\algorithmicensure{\textbf{Output:}}
\renewcommand{\thealgorithm}{} % don't number algorithms

%\usepackage[usenames,dvipsnames]{xcolor} % loaded in document class
\usepackage{tikz}
\usetikzlibrary{shapes}

% colors
\definecolor{dkred}{rgb}{.5,0,0}
\definecolor{dark-gray}{gray}{0.3}
\definecolor{dkgray}{rgb}{.4,.4,.4}
\definecolor{gray}{rgb}{.4,.4,.4}
\definecolor{grey}{rgb}{.4,.4,.4}
\definecolor{dkblue}{rgb}{0,0,.5}
\definecolor{medblue}{rgb}{0,0,.75}
\definecolor{rust}{rgb}{0.5,0.1,0.1}

% emphasis
\newcommand{\require}[1]{\textcolor{red}{#1}}
\newcommand{\red}[1]{\textcolor{red}{#1}}
\newcommand{\good}[1]{\textcolor{blue}{#1}}
\newcommand{\bad}[1]{\textcolor{dkred}{#1}}
\newcommand{\bred}[1]{\textcolor{red}{\textbf{#1}}}

% notes
\newcommand{\mnote}[1]{\textcolor{purple}{\textbf{[MU: #1]}}}

% formatting
\newcommand{\comment}[1]{\setstretch{0.5}\vfill\textcolor{grey}{\scriptsize{Comments: #1}}\\}
\newcommand{\source}[1]{\vfill\textcolor{grey}{\scriptsize{Source: #1}}}
\newcommand{\pitem}{\pause \item}

%%%%%% from the paper

\newcommand{\beas}{\begin{eqnarray*}}
\newcommand{\eeas}{\end{eqnarray*}}
\newcommand{\bea}{\begin{eqnarray}}
\newcommand{\eea}{\end{eqnarray}}
\newcommand{\beq}{\begin{equation}}
\newcommand{\eeq}{\end{equation}}
\newcommand{\bit}{\begin{itemize}}
\newcommand{\eit}{\end{itemize}}
\newcommand{\ben}{\begin{enumerate}}
\newcommand{\een}{\end{enumerate}}

\newcommand{\ba}{\begin{array}}
\newcommand{\ea}{\end{array}}
\newcommand{\bbm}{\begin{bmatrix}}
\newcommand{\ebm}{\end{bmatrix}}

% text abbrevs

%%% Modern composition rules set these Latin abbreviations in roman text.
%%% Exempli gratia and id est have a comma afterward.
%%% cf. The Chicago Manual of Style.

\newcommand{\cf}{cf.}
\newcommand{\eg}{e.g.}
\newcommand{\ie}{i.e.}
\newcommand{\etc}{etc.}

\newcommand{\ones}{\mathbf 1}

% std math stuff
\newcommand{\reals}{{\mbox{\bf R}}}
\newcommand{\integers}{{\mbox{\bf Z}}}
\newcommand{\posint}{{\mbox{\bf N}}}
\newcommand{\eqbydef}{\mathrel{\stackrel{\Delta}{=}}}
\newcommand{\complex}{{\mbox{\bf C}}}
\newcommand{\sym}{{\mbox{\bf S}}}  % symmetric matrices

% lin alg stuff
\newcommand{\Span}{\mathop{\bf span}}
\newcommand{\range}{\mathop{\bf range}}
\newcommand{\rank}{\mathop{\bf rank}}
\newcommand{\nullspace}{{\mathop {\bf null}}}
\newcommand{\tr}{\mathop{\bf tr}}
\newcommand{\Tr}{\mathop{\bf tr}}
\newcommand{\diag}{\mathop{\bf diag}}
\newcommand{\lambdamax}{{\lambda_{\rm max}}}
\newcommand{\lambdamin}{\lambda_{\rm min}}
\newcommand{\ms}{\mathop{\bf MaxSingVec}}

% norm
\newcommand{\twonorm}[1]{\left\|#1\right\|_2}
\newcommand{\norm}[1]{\left\|#1\right\|}
\newcommand{\fronorm}[1]{\left\|#1\right\|_{\mbox{\tiny{F}}}}
\newcommand{\opnorm}[1]{\left\|#1\right\|_{\mbox{\tiny{\textup{op}}}}}
\newcommand{\nucnorm}[1]{\left\|#1\right\|_*}
\newcommand{\arbnorm}[1]{\left\|#1\right\|_\alpha}
\newcommand{\darbnorm}[1]{\left\|#1\right\|^*_\alpha}
% probability stuff
\newcommand{\Expect}{\mathop{\bf E{}}}
\newcommand{\Prob}{\mathop{\bf Prob}}
\newcommand{\erf}{\mathop{\bf erf}}
\newcommand{\sign}{\mathop{\bf sign}}

% convexity & optimization stuff
\newcommand{\Co}{{\mathop {\bf Co}}}
\newcommand{\co}{{\mathop {\bf Co}}}
\newcommand{\Var}{\mathop{\bf var{}}}
\newcommand{\dist}{\mathop{\bf dist{}}}
\newcommand{\Ltwo}{{\bf L}_2}
%\renewcommand{\QED}{~~\rule[-1pt]{8pt}{8pt}}\def\qed{\QED}
\newcommand{\approxleq}{\mathrel{\smash{\makebox[0pt][l]{\raisebox{-3.4pt}{\small$\sim$}}}{\raisebox{1.1pt}{$<$}}}}
\newcommand{\argmin}{\mathop{\rm argmin}}
\newcommand{\epi}{\mathop{\bf epi}}
%\newcommand{\hypo}{\mathop{\bf hypo}}
\newcommand{\var}{\mathop{\bf var}}
\newcommand{\Card}{\mathop{\bf card}}
\newcommand{\vol}{\mathop{\bf vol}}
\newcommand{\card}{\mathop{\bf card}}
\newcommand{\conv}{\mathop{\bf conv}}
\newcommand{\dom}{\mathop{\bf dom}}
\newcommand{\aff}{\mathop{\bf aff}}
\newcommand{\cl}{\mathop{\bf cl}}
\newcommand{\Angle}{\mathop{\bf angle}}
\newcommand{\intr}{\mathop{\bf int}}
\newcommand{\relint}{\mathop{\bf rel int}}
\newcommand{\bd}{\mathop{\bf bd}}
\newcommand{\vect}{\mathop{\bf vec}}
\newcommand{\dsp}{\displaystyle}
\newcommand{\foequal}{\simeq}
\newcommand{\VOL}{{\mbox{\bf vol}}}
\newcommand{\argmax}{\mathop{\rm argmax}}
\newcommand{\xopt}{x^{\rm opt}}
\newcommand{\prox}{{\bf prox}}
\newcommand{\conj}{{\bf conj}}
\newcommand{\update}{\mathop{\bf update}}
\newcommand{\indicator}{{\bf 1}}

% for statistics
\newcommand{\huber}{\mathop{\bf huber}}
\newcommand{\sort}{\mathop{\bf sort}}
\newcommand{\round}{\mathop{\bf round}}


% for prettiness
\newcommand{\half}{(1/2)}

% for tucker apx talk
\newcommand{\M}[1]{\mathbf{#1}}
\newcommand{\E}{\mathbb{E}}
\newcommand{\V}[1]{\mathbf{#1}}
\newcommand{\T}[2][]{\boldsymbol{#1\mathscr{\MakeUppercase{#2}}}}
\newcommand{\vc}{\mathop{\mathbf{vec}}}
\newcommand{\cov}{\mbox{Cov}}


\begin{document}
\maketitle




\section{Introduction}
\subsection{Notation}
Our paper follows the notation of \cite{kolda2009tensor}. We denote the \textit{scalar}, \textit{vector}, \textit{matrix}, and \textit{tensor}, respectively by lowercase letters, ($x$) boldface lowercase letters ($\mathbf{x}$)  boldface capital letters  ($\mathbf{X}$)  and Euler script letters ($\mathscr{X}$). For matrix $\mathbf{X} \in \mathbb{R}^{m \times n}$, $\mathbf{X}^\dag \in \mathbb{R}^{n \times m}$ denotes its \textit{Moore-Penrose pseudoinverse}. In particular, $\mathbf{X}^\dag = (\mathbf{X}^\top \mathbf{X})^{-1}\mathbf{X}^T$, if $m \geq n$ and $\mathbf{X}$ has full column rank; $\mathbf{X}^\dag = \mathbf{X}^T(\mathbf{XX}^T)^{-1}$, if $m < n$ and $\mathbf{X}$ has full row rank. We let $[N]$ be the set containing $1,\dots, N$.

For a tensor $\mathscr{X} \in \mathbb{R}^{I_1 \times \cdots \times I_N}$, its \textit{mode} or \textit{order} is the number of dimensions $N$. If $I = I_1 = \cdots I_N$, we denote $\mathbb{R}^{I_1 \times \cdots \times I_N}$ as $\mathbb{R}^{I^N}$. The inner product of two tensors $\mathscr{X}, \mathscr{Y}$ is defined as $\langle \mathscr{X}, \mathscr{Y}\rangle = \sum_{i_1=1}^{I_1}\cdots \sum_{i_N=1}^{I_n} \mathscr{X}_{i_1\dots i_N}\mathscr{Y}_{i_1\dots i_N}$. The \textit{Frobenius norm} of $\mathscr{X}$ is denoted by $\|\mathscr{X}\|_F = \sqrt{\langle \mathscr{X}, \mathscr{X}\rangle}$. Let $\bar{I} = \Pi_{j = 1}^N I_j $ and $I_{(-n)} = \Pi_{j \neq n} I_j $. We denote the \textit{mode-n unfolding} of $\mathscr{X}$ as $\mathbf{X}^{(n)} \in \mathbb{R}^{I_n \times I_{(-n)}}$ and the \textit{mode-n rank} as the rank of the mode-n unfolding. We define the \textit{rank} of  $\mathscr{X}$ as $\mathbf{r}(\mathscr{X}) = (r_1,\dots, r_N)$ if its \textit{mode-n rank} is $r_n$ for all $n\in [N]$. The tensor with all entries equal to 0 except for the entries with the same indices is called a \textit{superdiagonal} tensor.\par

 $\mathscr{X} \times_n \mathbf{U}$ denotes the \textit{mode-n (matrix) product} of $\mathscr{X}$ with $\mathbf{U} \in \mathbb{R}^{J \times I_n}$, with size $I_1 \times \cdots \times I_{n-1} \times J \times I_{n+1} \times \cdots \times I_N$, that is:
\begin{equation}
\mathscr{G} = \mathscr{X} \times_n \mathbf{U} \; \iff \; \mathbf{G}^{(n)} = \mathbf{U}\mathbf{X}^{(n)}.
\end{equation}

The \textit{Kronecker product} of two matrices $\mathbf{A}\in R^{m\times n},\mathbf{B} \in R^{p\times q}$ is denoted by $\mathbf{A} \otimes \mathbf{B}$ which is of size $mp\times nq$ defined as 
\begin{equation}
\mathbf{A} \otimes \mathbf{B} = \left[
\begin{array}{ccc}
  A_{11}B   & \cdots & A_{1n}B \\
  \vdots & \ddots & \vdots \\
  A_{m1}B & \cdots &   A_{mn}B
\end{array}
\right].
\end{equation}
We let $\mathbf{X} \odot \mathbf{Y}$ denotes the \textit{Khatri-Rao product}, $\mathbf{A} \in \mathbb{R}^{I \times K}, \mathbf{B} \in \mathbb{R}^{J \times K}$, i.e. the "matching columnwise" Kronecker product. The resulting matrix of size $(IJ) \times K$ is given by: 
\begin{equation}\label{khatri-rao}
\mathbf{A} \odot \mathbf{B} = [\mathbf{a}_1 \otimes \mathbf{b}_1, \dots, \mathbf{a}_K \otimes \mathbf{b}_K]
\end{equation}
For convenience, we denote the Khatri-Rao product matching row-wise as $\odot^\top$, that is, $\mathbf{A} \odot^\top \mathbf{B} = (\mathbf{A}^\top \odot \mathbf{B}^\top)^\top $, where $\mathbf{A} \in \mathbb{R}^{K \times I}, \mathbf{B} \in \mathbb{R}^{K \times J}$. We write column of matrix $\mathbf{A} = [a_1\mid \cdots \mid a_k]$






\section{Structured Random Projection Preserving Column Space}


\end{document}

\end{document}
