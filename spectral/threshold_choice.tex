 
\subsection{Choice of Tuning Parameters}\label{sec:tuning-parameter}
At any Fourier frequency $\omega_j$, we need to choose two tuning parameters for our method - (i) the smoothing span $2m+1$, and (ii) the threshold level $\lambda$. In this work, we select a single smoothing span for all the frequencies, but choose the threshold level separately for each frequency.

% The first one is the smoothing bandwidth $2m + 1$. 
%The smoothing method is running average. {\color{red} Provide formula, or refer to equation number in introduction section.}
The smoothing span plays the role of ``effective sample size'' in estimating $f(\omega_j)$. Recall that 
\begin{equation}
    \hat{f}(\omega_j; m) = \frac{1}{2\pi (2m+1)} \sum_{|k|\le m} I(\omega_{j+k}). \nonumber
\end{equation}
In classical asymptotics ($p$ fixed, $n \rightarrow \infty$) and the Kolmogorov asymptotics ($p \rightarrow \infty, n \rightarrow \infty, p^2/n \rightarrow 0$) \citep{brockwell2013time, bohm2009shrinkage}, it is shown that for $\hat{f}(\omega_j; m)$ to be consistent, $m$  (depending on $n$) must go to infinity and $m/n \rightarrow 0$ as $n\rightarrow \infty$. Our non-asymptotic analysis in Section \ref{sec:theory} suggests that $m/[n\Omega_n(f)]  \rightarrow 0$, where $\Omega_n(f)$, defined as $\max_{r,s} \sum_{\ell=-n}^n |\ell| |\Gamma_{rs}(\ell)|$ is a measure of temporal dependence in the time series. For our numerical and real data applications, we choose $m$ in the order of  $\sqrt{n}$, with smaller values of $m$ for processes with stronger temporal dependence and larger $\Omega_n(f)$. A more data-driven approach along the line of \cite{ombao2001smoothing} and \citet{fiecas2014datadriven} can be designed with suitable modification to account for high-dimensionality, although we do not pursue this direction in this work.




The second tuning parameter is the threshold value. Unlike the shrinkage estimators of spectral density matrices, finding  asymptotically optimal plug-in estimators for threshold level is challenging due to the non-smooth nature of thresholding operators. For covariance estimation from  i.i.d. data using thresholding, a sample-splitting method proposed in \citet{bickel2008covariance} or its variants are normally employed. In this method, the entire sample is split into two sub-samples, and the Frobenius norm difference between thresholded estimation in one sub-sample and regular sample covariance in the other sub-sample is compared for different levels of threshold. The entire exercise is repeated $N$ times and the level of threshold minimizing the average Frobenius norm difference is selected as the threshold.

This approach is not directly amenable to spectral density estimation since for any two given sub-sample sizes, only $N=1$ split is possible maintaining the temporal ordering. However, the periodograms at different positive Fourier frequencies $\omega_j \in F_n, \omega_j \ge 0$, are asymptotically independent. This suggests an analogous sample-splitting algorithm can be designed in the frequency domain. With this heuristic, we propose the following algorithm.

For each frequency $j$ $\in \{1, \cdots, [n/2]\}$, we randomly split the periodograms in $\{j-m ,\cdots,j+m\}$ into two sub-samples $J_1, J_2$ of size $m_1$ and $m_2$, where $|m_1 - m_2 | \le 1$. Since $I(\omega_{-k}) = I(\omega_{k})$, we keep $I(\omega_k)$ and $I(-\omega_k)$ in the same sub-sample. Then, for every $\lambda$ on a finite grid of possible threshold choices $\mathcal{L}$, we calculate the squared Frobenius norm of the difference between thresholded averaged periodogram on $J_1$, viz.,  $\hat{f}_{1}(\omega_j)$,  and averaged periodogram $\hat{f}_{2}(\omega_j)$ on $J_2$. This exercise is repeated $N$ times and the threshold $\lambda \in \mathcal{L}$ minimizing squared Frobenius norm is selected as $\hat{\lambda}_j$ for frequency $\omega_j$. A complete description is provided in Algorithm \ref{al1}.






