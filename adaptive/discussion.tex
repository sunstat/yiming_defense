\section{Discussion}\label{sec:discussion}
We proposed hard thresholding and generalized thresholding of averaged periodogram for estimation of high-dimensional spectral density matrices of stable Gaussian time series and linear processes with errors having potentially heavier tails than Gaussian. Under high-dimensional regime $\log p /n \rightarrow 0$, we established consistency of the above estimation procedures when the true spectral densities are weakly sparse. At the core of our technical results lie concentration inequalities of  complex quadratic forms of temporally dependent, high-dimensional random vectors, which were used to derive finite sample deviation of averaged periodograms around their expectation. These results  are of independent interest and are potentially useful in other problems involving high-dimensional spectral density. In our next steps, we plan to extend the theoretical analyses to more general adaptive thresholding methods \citep{cai2011adaptive}, which will explicitly account for heterogeneity in the strengths of cross-spectral association across different pairs of time series and different frequency bands. We also plan to develop estimation and inference procedures for high-dimensional partial coherence at different frequencies.

Another direction of potential interest is to develop thresholding strategies that incorporate  information on different brain regions and prior biological knowledge on brain networks. Dynamic functional connectivity of brain networks is known to play important roles behind progression of neurodegenerative diseases. A common approach to build such networks is using coherence measures of Fourier or wavelet transform of multi-channel fMRI/EEG/MEG signals and thresholding small entries of zero. Selection of threshold level that represents heterogeneous modular structure of human brain has been a topic of active research \citep{bordier2017graph}. We expect that more sophisticated thresholding methods, building up on universal and adaptive thresholds and incorporating prior neuroscientific knowledge, will be potentially useful in  data-driven discovery of scientifically and clinically relevant connectivity patterns in human brain.

\section*{Acknowledgements} The authors wish to thank Pratik Mukherjee for providing and Keith Jamison for pre-processing the TBI patient MRI data.  SB was supported by NSF award (DMS-1812128) and AK was supported by a Kellen Foundation Fellowship and the NIH (R21 NS104634-01 and R01 NS102646-01A1).