\section{Proof for Bias Bounding}
\subsection{Proof for Lemma \ref{lemma:bound_deviation}}
\begin{proof}
We will show proof for real part only and with same argument, we can finish the proof for imaginary part. By triangular inequality, 
\begin{equation}
\label{eq:expectation_bias_decomposition}
\begin{aligned}
&\left|\mathbb{E}[{\bf Re}(g_{rs}(\omega_j))] - {\bf Re}(f_{rs}(\omega_j))\right| \\
&\le \left|\mathbb{E}[{\bf Re}(g_{rs}(\omega_j))] - \mathbb{E}[{\bf Re}(\hat{f}_{rs}(\omega_j))] \right| + \left|\mathbb{E}[{\bf Re}(\hat{f}_{rs}(\omega_j))] - {\bf Re}(f_{rs}(\omega_j))\right|,
\end{aligned}
\end{equation}
where $\hat{f}(\omega_j)$ is the usual smoothed estimator. \par 
It is not hard to see that 
\begin{equation}
\left|\mathbb{E}[{\bf Re}(g_{rs}(\omega_j))] - \mathbb{E}[{\bf Re}(\hat{f}_{rs}(\omega_j))] \right| \le \max_{k\in \mathcal{B}_j} \left| e_r^\top   (H(\omega_k) - I(\omega_k)) e_s\right|. 
\end{equation}
Now for any unit vector $u,v$, 
\begin{equation}
\begin{aligned}
\label{eq:ada_periodogram_real_part_dif}
&\mathbb{E} \left [u^\top  {\bf Re}(H(\omega_j))v - u^\top  {\bf Re}(I(\omega_j))v\right] = \frac{1}{2\pi} \mathbb{E}\left[u^\top  (\mathcal{X}^\top  c_jc_j^\top   \mathcal{X} -  \mathcal{X}^\top  s_js_j^\top   \mathcal{X})v\right]\\
&=\frac{1}{2\pi} \mathbb{E}\left[c_j^\top   \mathcal{X}vu^\top  \mathcal{X}^\top   c_j - s_j^\top   \mathcal{X}vu^\top   \mathcal{X}^\top  s_j\right].
\end{aligned}
\end{equation}
Let $\Sigma^{v,u} = \cov(\mathcal{X}v, \mathcal{X}u)$, then $\Sigma^{v,u}$ is a toeplitz matrix with element as 
\begin{equation}
\Sigma^{v,u}_{k,q} = u^\top  \Gamma(q-k)v.
\end{equation}
Then by first part of lemma \ref{lemma: bound_toeplitz}, we know $\left|c_j^\top  \Sigma^{v,u}c_j-s_j^\top  \Sigma^{v,u}s_j\right|\le \frac{\Omega(\Sigma^{v,u})}{2\pi n}$. Besides, $\Omega(\Sigma^{v,u}) \le \Omega_n(f_\mathcal{X})$ since  $\left|u^\top  \Gamma(q-k)v\right| \le \|\Gamma(q-k) \| $. Hence,we could bound \eqref{eq:ada_periodogram_real_part_dif} with $\frac{1}{2\pi} \Omega_n(f_\mathcal{X})$. \par 
Besides, Lemma A.4 in \cite{sun2018large} shows that the second term in \eqref{eq:expectation_bias_decomposition} is bounded by $\frac{m}{n}\Omega_n(f_\mathcal{X}) + \frac{1}{2\pi}\left(\frac{\Omega_n(f_\mathcal{X})}{n}+L_n(f_\mathcal{X})\right)$. Combining these two bounds, we complete the proof. 
 \end{proof}
 
\subsection{Proof for Lemma \ref{lemma:theta_bias}} 
\begin{proof}
We will only present the proof for building deviation bound for  $\hat{\theta}^{(r)}_{j,rs}$ and the same argument applies to achieve deviation bound for  $\hat{\theta}^{(i)}_{j,rs}$.
For $j\in F_n$, let $y_j$(we omit $r,s$ from notation for simplicity) be the vector of length $m$, composed by  ${\bf Re}(H(\omega_k)_{rs}), k\in \mathcal{B}_j$. Then 
\begin{equation}
\begin{aligned}
&\mathbb{E} \hat{\theta}^{(r)}_{j,rs} = \frac{1}{m-1}\mathbb{E}\left\{y_j^\top   \left[I-\frac{1}{m}11^\top   \right]y_j\right\} = \frac{1}{m-1}\rm{Tr}\left(\left[I-\frac{1}{m}11^\top   \right] D\right) \\
&= \frac{1}{m-1}\left\{\sum_{q\in B^j_m} \left(1-\frac{1}{m}\right)D(q,q) + \frac{1}{m} \sum_{q_1\neq q_2, \in B^j_m}D(q_1,q_2)\right\}, 
\end{aligned}
\end{equation}
where 
\begin{equation}
D(q_1,q_2) = \cov({\bf Re}(H_{rs}(\omega_{q_1})), {\bf Re}(H_{rs}(\omega_{q_2}))), q_1,q_2 \in \mathcal{B}_j.
\end{equation}
If $D(q,q)$ share the same value as $\theta^r_{rs}$ and $D(q_1,q_2)=0$ then $\hat{\theta}^{(r)}_{j,rs}$ is an unbiased estimator. So in this proof we bound  
$|D(q,q) - \theta^r_{rs}|$ and $|D(q_1,q_2)|$.\\[0.2cm]
{\bf Bound $|D(q,q) - \theta^{(r)}_{j, rs}|$:}\\
We first bound  $|D(q,q)-\theta^{(r)}_{j, rs}|, q\in \mathcal{B}_j$. 
For $e_r, e_s$, applying results of Gaussian fourth moments in section \ref{sec:technical_lemmas}, 
\begin{equation}
\label{eq: dif_liminting_var}
\begin{aligned}
&|D(q,q)-\theta^{(r)}_{j, rs}| \\
&=  \left|4\var(\langle \mathcal{X}^\top   c_q,e_r\rangle \langle \mathcal{\mathcal{X}}^\top   c_q,e_s\rangle) - \var(e_r^\top  {\bf Re}(H_{\infty}(\omega_j))e_s)\right| \\
&\le 4|\var(\langle \mathcal{\mathcal{X}}^\top   c_q,e_r\rangle)\var{\langle \mathcal{\mathcal{X}}^\top   c_q,e_s\rangle)} - \var(\langle {\bf Re}(d_{\infty, j}),e_r\rangle)\var{\langle {\bf Re}(d_{\infty, j}),e_s\rangle)}| \\
&+ 4|\mathbb{E}^2(\langle \mathcal{\mathcal{X}}^\top   c_q,e_r\rangle \langle \mathcal{\mathcal{X}}^\top  c_q,e_s\rangle) )-\mathbb{E}^2(\langle {\bf Re}(d_{\infty, j}),e_r\rangle \langle {\bf Re}(d_{\infty, j}),e_s\rangle) )|
\end{aligned}
\end{equation}
As shown before, $2\var(\langle {\bf Re}(d_{\infty, j}), e_r\rangle) = f_{rr}(\omega_j)$,  then 
\begin{equation}
\label{eq:help_bound1}
\begin{aligned}
&|\var(\langle \mathcal{X}^\top c_q,e_r\rangle) -  \var(\langle {\bf Re}(d_{\infty, j}),e_r\rangle)| =\frac{1}{2} \left|e_r^\top \mathbb{E}[{\bf Re}(H(\omega_q))]e_r - e_r^\top {\bf Re}(f(\omega_j))e_r\right|\\
& \le \frac{1}{2}|u^\top \mathbb{E}[{\bf Re}(H(\omega_q))]e_r - e_r^\top  \mathbb{E}[{\bf Re}(I(\omega_q))]e_r|+\frac{1}{2}|e_r^\top\mathbb{E}[{\bf Re}(I(\omega_q))]e_r-e_r^\top  {\bf Re}\mathbb{E}[(I(\omega_j))]e_r|\\
&+ \frac{1}{2}|e_r^\top\mathbb{E}[I(\omega_j)]e_r - e_r^\top  f(\omega_j)e_r|\\
&\le  \frac{\Omega_n}{4n\pi}+\frac{\sqrt{2}}{4\pi}\frac{|j-k|\Omega_n}{n}+ \frac{1}{4\pi}\left(\frac{\Omega_n}{n} + L_n(f_\mathcal{X})\right)\\
\end{aligned}
\end{equation}
where the first part of last line is from same technique used in lemma \ref{lemma:bound_deviation} and the second and third parts are from \cite{sun2018large}. 
Therefore, 
\begin{equation}
\begin{aligned}
&4|\var(\langle \mathcal{X}^\top  c_q,e_r\rangle)\var({\langle \mathcal{X}^\top   c_q,e_s\rangle)} - \var(\langle {\bf Re}(d_{\infty, j}) ,e_r\rangle)\var(\langle {\bf Re}(d_{\infty, j}), e_s\rangle)| \\
& \le |4\var(\langle \mathcal{X}^\top c_q,e_r\rangle)\var(\langle \mathcal{X}^\top  c_q,e_s\rangle)- 2\var(\langle \mathcal{X}^\top  c_q,e_r\rangle)2\var(\langle d^{r}_{\infty, j},e_s\rangle)| \\
&+|2\var(\langle \mathcal{X}^\top  c_q,e_r\rangle)2\var(\langle d^{r}_{\infty, j},e_s\rangle)- 2\var(\langle d^{r}_{\infty, j},e_r\rangle)2\var(\langle d^{r}_{\infty, j},e_s\rangle)|\\
&\le 2f_{rr}(\omega_j) \delta_1+\delta_1^2
\end{aligned}
\end{equation}
where $\delta_1 =  \frac{\Omega_n}{2n\pi}+\frac{\sqrt{2}}{2\pi}\frac{m\Omega_n}{n}+ \frac{1}{2\pi}\left(\frac{\Omega_n}{n} + L_n\right)$. Here we assume $f_{rr}(\omega_j)\geq f_{ss}(\omega_j)$ without loss of generality. 
Now we bound last line in \eqref{eq: dif_liminting_var}, noticing 
\[
2\mathbb{E}[\langle \mathcal{X}^\top  c_{q}, e_r\rangle\langle \mathcal{X}^\top  c_{q}, e_s\rangle] = e_r^\top  f(\omega_q)e_s
\], 
\begin{equation}
\begin{aligned}
&4|\mathbb{E}^2(\langle \mathcal{X}^\top  c_q,e_r\rangle \langle \mathcal{X}^\top  c_q,e_s\rangle) )-\mathbb{E}^2(\langle d^{r}_{\infty, j},e_r\rangle \langle d^{r}_{\infty, j},e_s\rangle) )|\\
&\le | e_r^\top  \mathbb{E}[{\bf Re}(H(\omega_q)]e_s+e_r^\top  {\bf Re}(f(\omega_j))e_s||\mathbb{E} (e_r^\top  {\bf Re}(H(\omega_q))e_s-e_r^\top  {\bf Re}(f(\omega_j))e_s|\\
&\le 2f_{rr}(\omega_j)\delta_2+\delta_2^2,
\end{aligned}
\end{equation}
where $\delta_2= \frac{\Omega_n}{2n\pi}+ \frac{1}{2\pi}\left(\frac{\Omega_n}{n} + L_n(f_X)\right)$. Here we use the fact that for $u, v\in E_p$, 
\begin{equation}
\begin{aligned}
&|\mathbb{E} [2(\langle d^{r}_{\infty, j},e_r\rangle \langle d^{r}_{\infty, j},v\rangle)]|  = |e_r^\top  f(\omega_j)e_s|\\
&\le \sqrt{(e_r^\top  f(\omega_j)e_r)(e_s^\top  f(\omega_j)e_s)}\le f_{rr}(\omega_j), 
\end{aligned}
\end{equation}
and same techniques to get bound as \eqref{eq:help_bound1}. \\[0.2cm]
{\bf Bound $|D(q_1,q_2)|$:}\\
For any unit vector $u,v$, applying summary of Gaussian fourth moments \ref{subsec: gaussian_fourth_moments}, 
\begin{equation}
\begin{aligned}
&4|\cov\left(\langle \mathcal{X}^\top  c_{q_1}, u \rangle  \langle \mathcal{X}^\top  c_{q_1}, v \rangle ,  \langle \mathcal{X}^\top  c_{q_2}, u \rangle  \langle \mathcal{X}^\top  c_{q_2}, v \rangle\right)| \\
&= 4\mathbb{E} \left[ \langle \mathcal{X}^\top  c_{q_1}, u \rangle  \langle \mathcal{X}^\top  c_{q_2}, u \rangle \right] \mathbb{E} \left[ \langle \mathcal{X}^\top  c_{q_1}, v \rangle  \langle \mathcal{X}^\top  c_{q_2}, v \rangle \right]\\
&+4\mathbb{E} \left[ \langle \mathcal{X}^\top  c_{q_1}, u \rangle  \langle \mathcal{X}^\top  c_{q_2}, v \rangle \right]\mathbb{E} \left[ \langle \mathcal{X}^\top  c_{q_1}, v \rangle  \langle \mathcal{X}^\top  c_{q_2}, u \rangle \right]
\end{aligned}
\end{equation}
We will show each of these four terms well bounded. For $\mathbb{E} \left[ \langle \mathcal{X}^\top  c_{q_1}, u \rangle  \langle \mathcal{X}^\top  c_{q_2}, u \rangle \right]$, 
\begin{equation}
\begin{aligned}
&|\mathbb{E} \left[ \langle \mathcal{X}^\top  c_{q_1}, u \rangle  \langle \mathcal{X}^\top  c_{q_2}, u \rangle\right]| = |\mathbb{E} c_{q_1}^\top   \Sigma^{u, u} c_{q_1}| \\
& \le \frac{\Omega(\Sigma^{u, u})}{2n\pi} \le \frac{\Omega_n}{2n\pi},
\end{aligned}
\end{equation}
where $\Sigma^{u, u}_{rs} = u^\top   \Gamma(s-r)u$ and the last inequality comes from the same argument we used in lemma \ref{lemma:bound_deviation}.
With almost same argument, we could show 
\begin{equation}
\begin{aligned}
&\max\left(\left|\mathbb{E} \left[ \langle \mathcal{X}^\top  c_{q_1}, u \rangle  \langle \mathcal{X}^\top  c_{q_2}, u \rangle \right]\right|,  \left|\mathbb{E} \left[ \langle \mathcal{X}^\top  c_{q_1}, v \rangle  \langle \mathcal{X}^\top  c_{q_2}, v \rangle \right]\right|\right.\\
&\left. \left|\mathbb{E} \left[ \langle \mathcal{X}^\top  c_{q_1}, u \rangle  \langle \mathcal{X}^\top  c_{q_2}, v \rangle \right]\right|, \left|\mathbb{E} \left[ \langle \mathcal{X}^\top  c_{q_1}, v \rangle  \langle \mathcal{X}^\top  c_{q_2}, u \rangle \right]\right|\right) \le \frac{\Omega_n}{2\pi n}
\end{aligned}
\end{equation}
Taking $u= e_{q_1}, v= e_{q_2}$, we show that 
\begin{equation}
D(q_1,q_2) \le \frac{\Omega_n^2}{\pi^2n^2}. 
\end{equation}

Above all, $\left |\mathbb{E} \hat{\theta}^{(r)}_{j,(r,s)}-\var({\bf Re}(H_{r,s}(\omega_j)))\right| \le 2\max(f_{rr}(\omega_j),f_{ss}(\omega_j))(\delta_1+\delta_2)+\delta_1^2+\delta_2^2+\frac{\Omega_n^2}{\pi^2n^2}$. Then by same argument, we could get the proof for imaginary part. 
\end{proof}

\subsection{Proof for Lemma \ref{lemma: variance_ratio_error}}
\begin{proof}
We will only provide the proof for real part. 
\begin{equation}
\begin{aligned}
& \Var(g_{rs}(\omega_j)) = \Var\left(\frac{1}{m}\sum_{q\in \mathcal{B}_j} H_{rs}(\omega_q)\right) \\
& = \sum_{q\in \mathcal{B}_j} \frac{1}{m^2} (\Var(H_{rs}(\omega_q))) \\
& + \frac{1}{m^2} \sum_{q_1\neq q_2 \in \mathcal{B}_j} \cov[H_{rs}(\omega_{q_1}), H_{rs}(\omega_{q_2})].  
\end{aligned}
\end{equation}
From proof in lemma \ref{lemma:theta_bias}, noticing $\delta_1\ge \delta_2$, 
\[
\left|\Var(H_{rs}(\omega_q)) - \theta^{(r)}_{j, rs} \right| \le 4 \max\{f_{rr}(\omega_j), f_{ss}(\omega_j)\} \delta_1 + 2\delta_1^2, 
\]
and 
\[
\cov (H_{rs}(\omega_{q_1}), H_{rs}(\omega_{q_2})) \le \frac{\Omega_n^2}{\pi^2n^2}. 
\]
Combining these two and the fact that $\theta_{j, rs}^r\ge f_{rr}(\omega_j)f_{ss}(\omega_j)$ and assumption $\min_{r=1}^p f_{rr}(\omega_j)\ge \phi_0$ we can get the result. 
\end{proof}



