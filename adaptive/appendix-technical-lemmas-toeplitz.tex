\section{Technical Results for Toeplitz Matrixz}
\begin{lem}
\label{lemma:sin_cos_seq_sum}
\begin{equation}
\begin{aligned}
&\sum_{k=1}^n \cos(a+kx) = \cfrac{\sin((1/2+n)x+a)-\sin(x/2+a)}{2\sin(x/2)}\\
&\sum_{k=1}^n \sin(a+kx) = \cfrac{\cos(x/2+a)-\cos((1/2+n)x+a)}{2\sin(x/2)}
\end{aligned}
\end{equation}
\begin{proof}
Proof is from element math which can be found in \cite{zygmund2002trigonometric}. 
\end{proof}
\end{lem}


\begin{lem}
\label{lemma: bound_toeplitz}
Let $M$ be a toeplitz matrix and $M_{p,q} = g(q-p)$ for some function $g(\cdot)$. Now we claim following results with definition of definition of $\cos, \sin$ sequence in \ref{eq:cos_sin_coef}, for $j\neq k$ and $\omega_j \neq 0, \pi$
\begin{equation*}
\max\{|c_j^\top M c_j - s_j^\top M s_j|, |s_j^\top M c_j + c_j^\top M s_j|\}\le \frac{\Omega(M)}{2\pi n}.
\end{equation*}
and 
\begin{equation*}
\max\left\{|c_j^\top M s_k|, |c_j^\top M c_k|, |s_j^\top M s_k|, |s_j^\top M c_k|\right\} \le \frac{\Omega(M)}{2\pi n},
\end{equation*}
where linear operator $\Omega$ on toeplitz matrix is defined as 
\begin{equation}
\Omega(M) = \sum_{\ell = -(n-1)}^{(n-1)} |\ell| |g(\ell)|.   
\end{equation}
\begin{remark}
Lemma \ref{lemma:asy_dis_dft} claims $X^\top c_j$ and $X^\top s_j$ have same limiting marginal distribution, 
and the first part of this lemma, in fact could be used to quantify this similarity in finite sample through quantification of the difference in two covariance matrix
which we will show later. The second part could be used to quantify the rate of being asymptotically independence across different frequency claimed by lemma \ref{lemma:asy_dis_dft}. \par 
There is many literature showing similar results in the second part, but usually it points out vanishing rate for any frequency between $[-\pi, \pi]$ in an asymptotic sense which is different from my theory. None of them provides an explicit finite sampling bound at discrete Fourier frequencies and. 
\end{remark}
\begin{proof}
We shall only prove $|c_j^\top M c_j - s_j^\top M s_j|\le \frac{\Omega(M)}{2\pi n}$, then all the others could be proven with same techniques. %Using formula $\cos(x+y) = \cos x\cos y-\sin x\sin y $%
\begin{equation}
\label{eq:similarity_real_im}
\begin{aligned}
 &|c_j^\top M c_j - s_j^\top M s_j| \\
 &=\frac{1}{2\pi n}\left|\sum_{\ell=-(n-1)}^{(n-1)} \sum_{p=0}^{(n-1)-\left|\ell\right|} g(\ell)\left[ \cos(p\omega_j)\cos((p+\ell)\omega_j) - \sin(p\omega_j)\sin((p+\ell)\omega_j)\right]\right|\\
 &\le \frac{1}{2\pi n} \left|\sum_{\ell 
 = -(n-1)}^{n-1} \sum_{p=0}^{n-1} g(\ell)\left[ \cos(p\omega_j)\cos((p+\ell)\omega_j) - \sin(p\omega_j)\sin((p+\ell)\omega_j)  \right]\right|\\
 &+\frac{1}{2\pi n}\left|\sum_{\ell=-(n-1)}^{(n-1)} \sum_{p=(n-1)-\left|\ell\right|}^{n-1} g(\ell)\left[ \cos(p\omega_j)\cos((p+\ell)\omega_j) - \sin(p\omega_j)\sin((p+\ell)\omega_j)\right]\right| \\
 &\le \frac{1}{2\pi n}\left|\sum_{\ell 
 = -(n-1)}^{n-1} \sum_{p=0}^{n-1} g(\ell)\cos(2p\omega_j+\ell\omega_j) \right|+ \frac{1}{2\pi n}\sum_{\ell = -(n-1)}^{(n-1)} |\ell| |g(\ell)|.\\
 \end{aligned}
\end{equation}
By setting $a = (\ell-2)\omega_j$, $x = 2\omega_j$ in lemma \ref{lemma:sin_cos_seq_sum} and noticing $2n\omega_j = 4j\pi$, we get $\sum_{p=0}^{n-1} g(\ell)\cos(2p\omega_j+\ell\omega_j) = 0$ for any $\ell$. Therefore, \eqref{eq:similarity_real_im} leads to 
\begin{equation}
 |c_j^\top M c_j - s_j^\top M s_j| \le \frac{1}{2\pi n}\sum_{\ell = -(n-1)}^{(n-1)} |\ell| |g(\ell)| = \frac{\Omega(M)}{2\pi n}. 
\end{equation}
With similar techniques, we could prove the second part. 
\end{proof}
\end{lem}
